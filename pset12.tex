\documentclass[a4paper]{exam}


\usepackage[a4paper]{geometry}
\usepackage{amsfonts, amsmath, amsthm}
\usepackage[framemethod=TikZ]{mdframed}

\newcommand\Z{\ensuremath{\mathbb{Z}}}

\title{Problem Set 12: Strong Induction}
\author{CS/MATH 113 Discrete Mathematics}
\date{Spring 2024}

\boxedpoints

\printanswers

\begin{document}
\maketitle

In solving the problems below, note that mathematical induction is just a special case of strong induction.

\begin{questions}

\question Use strong induction to prove each of the statements given below.
  \begin{parts}
  \part Every integer \( n \geq 2 \) is either prime or a product of primes.\\
    \textit{Hint}: Consider the cases where \( k+1 \) is prime and where it is composite.
   \begin{solution}
    % Enter your solution here.
   \end{solution}

 \part \( n! \ge n \) for all integers, \( n \geq 1 \) using strong induction.
\begin{solution}
    % Enter your solution here.
  \end{solution}

\part  \( \Z^n \) is countable for all integers \( n \geq 1 \).\\
  \textit{Hint}: You can refer to known results and proofs to simplify your proof.\\
  \textit{Note}: Recall that \( \Z^n = \underbrace{\Z\times\Z\times\ldots\times\Z}_{n \text{ times}} \)
  \begin{solution}
    % Enter your solution here.
  \end{solution}
\end{parts}

\question The game of Mini-Nim is defined as follows: Some positive number of sticks are placed on the ground. Two players take turns removing one, two, or three sticks. The player to remove the last one loses.

  Use strong induction to show that: The second player has a winning strategy only if the number of sticks equals \( 4k + 1 \) for some integer \( k \ge 0 \).\\
  \textit{Note}: You will have to consider all cases for the number of sticks: \( 4k+1, 4k+2, 4k+3, \) and \( 4k+4 \), and show that Player 2 can win in one case, and that Player 1 can win in all the other cases.

  \begin{solution}
    % Enter your solution here.
  \end{solution}
  
  \question 
  Consider the following proof by strong induction that a class with \( n \geq 8 \) students can be divided into groups of 4 or 5.

  \begin{mdframed}
    \begin{proof} The proof is by strong induction.

      Let \( P(n) \) be the proposition that a class with \( n \) students can be divided into teams of 4 or 5.

      First, we prove that \( P(n) \) is true for \( n = 8, 9, \) or \( 10 \) by showing how to break classes of these sizes into groups of 4 or 5 students:
      \begin{align*}
8  &= 4 + 4 \\
9  &= 4 + 5 \\
10 &= 5 + 5 
      \end{align*}
Next, we must show that \( P(8), \ldots, P(n) \) imply \( P(n + 1) \) for all \( n \geq 10 \). Thus, we assume that \( P(8), \ldots, P(n) \) are all true and show how to divide up a class of \( n + 1 \) students into groups of 4 or 5. We first form one group of 4 students. Then we can divide the remaining \( n-3 \) students into groups of 4 or 5 by the assumption \( P(n-3) \). This proves \( P(n+1) \), and so the claim holds by induction.
\end{proof}
\end{mdframed}

a)
This proof contains a critical logical error. (In fact, the claim is false!) Identify the first sentence in the proof that does not follow and explain what went wrong.
  \begin{solution}
      
  \end{solution}
b)
Provide a correct strong induction proof that a class with \( n \geq 12 \) students can be divided into groups of 4 or 5.
  \begin{solution}
      
  \end{solution}

\end{questions}
\end{document}
%%% Local Variables:
%%% mode: latex
%%% TeX-master: t
%%% End: