\documentclass[a4paper]{exam}


\usepackage[a4paper]{geometry}
\usepackage{amsfonts, amsmath, amsthm}
\usepackage[framemethod=TikZ]{mdframed}

\newcommand\Z{\ensuremath{\mathbb{Z}}}
\newcommand\Q{\ensuremath{\mathbb{Q}}}

\title{Problem Set 12: Strong Induction}
\author{CS/MATH 113 Discrete Mathematics}
\date{Spring 2024}

\boxedpoints

\printanswers

\begin{document}
\maketitle

In solving the problems below, note that mathematical induction is just a special case of strong induction.

\begin{questions}

\question Use strong induction to prove each of the statements given below.
  \begin{parts}
  \part Every integer \( n \geq 2 \) is either prime or a product of primes.\\
    \textit{Hint}: Consider the cases where \( k+1 \) is prime and where it is composite.
   \begin{solution}
     The argument in the inductive step is along the following lines.

     If $k+1$ is prime, then the inductive step is trivially true.\\
     If $k+1$ is composite, it is the product of 2 smaller numbers, on which the IH applies and which are therefore, either prime of the product of primes. $k+1$ is therefore the product of primes.     
   \end{solution}

 \part \( n! \ge n \) for all integers, \( n \geq 1 \) using strong induction.
\begin{solution}
  The argument in the inductive step is along the following lines. Note that it uses mathematical induction, which is just a form of strong induction.

  $\qquad IH: k! \ge k$\\
  Multiplying with $k+1$ on both sides yields.

  $\qquad (k+1)k! \ge (k+1)k$\\
  The LHS is $(k+1)!$. On the RHS, as $k\ge 1$ on account of the domain, $(k+1)k \ge k$.
  
  $\qquad \implies (k+1)! \ge (k+1)$\\
  This is exactly $P(k+1)$.
  \end{solution}

\part  \( \Z^n \) is countable for all integers \( n \geq 1 \).\\
  \textit{Hint}: You can refer to known results and proofs to simplify your proof.\\
  \textit{Note}: Recall that \( \Z^n = \underbrace{\Z\times\Z\times\ldots\times\Z}_{n \text{ times}} \)
  \begin{solution}
    The basis step for $n=1$ is trivial.

    For the inductive step, $\Z^{k+1}$ can be written as the product of smaller powers of $\Z$, on which the IH applies, and which are therefore countable. It remains to show that the product of 2 countable sets is countable. This is a known result, as illustrated in the proof of the countability of \Q.
    % Enter your solution here.
  \end{solution}
\end{parts}

\question The game of Mini-Nim is defined as follows: Some positive number of sticks are placed on the ground. Two players take turns removing one, two, or three sticks. The player to remove the last one loses.

  Use strong induction to show that: The second player has a winning strategy only if the number of sticks equals \( 4k + 1 \) for some integer \( k \ge 0 \).\\
  \textit{Note}: You will have to consider all cases for the number of sticks: \( 4k+1, 4k+2, 4k+3, \) and \( 4k+4 \), and show that Player 2 can win in one case, and that Player 1 can win in all the other cases.

  \begin{solution}
    Any number of sticks at the start of the game can be written as $4k+1$, $4k+2$, $4k+3$, or $4k+4$ for some integer $k\ge0$.

    \underline{Case 1}: Proof by strong induction that Player 2 has a winning strategy with $4k+1$ initial sticks.

    For the basis step, $k=0$ and the starting number of sticks is $4(0) + 1 = 1$. Player 1 removes the only stick and so Player 2 wins.

    For the inductive step, the game starts with $4(k+1) + 1=4k+5$ sticks. Player 1 starts with a choice to remove 1, 2, or 3 sticks. If Player 1 removes 1 stick, Player 2 will remove 3 sticks in their turn, thus leaving Player 1 with $4k+1$ sticks. Similarly, if Player 1 removes 2 sticks, Player 2 will remove 2 sticks in response. And if Player 3 removes 3 sticks, Player 2 will remove 1 stick in response. In every case, Player 1 is left to move with a pile containing $4k+1$ sticks. By IH, we know that Player 2 can win from this stage.
    
    \underline{Case 2}: Proof by strong induction that Player 1 has a winning strategy with $4k+2$ initial sticks.

    For the basis step, $k=0$ and the starting number of sticks is $4(0) + 2 = 2$. Player 1 removes one stick, forcing Player 2 to remove the only remaining stick, and so Player 1 wins.

    For the inductive step, the game starts with $4(k+1) + 2=4k+6$ sticks. Player 1 starts with a choice to remove 1, 2, or 3 sticks. Player 1 removes 3 sticks, the maximum allowed in a turn. Player 2 may remove 1, 2, or 3 sticks in response, leaving Player 1 with a pile containing  $4k+2, 4k+1,$ or $4k$ sticks. Ih applies to all of these are Player 1 can win from here on.

    \underline{Cases 3, 4}: Can be argued similarly to Case 2.
    
  \end{solution}
  
  \question 
  Consider the following proof by strong induction that a class with \( n \geq 8 \) students can be divided into groups of 4 or 5.

  \begin{mdframed}
    \begin{proof} The proof is by strong induction.

      Let \( P(n) \) be the proposition that a class with \( n \) students can be divided into teams of 4 or 5.

      First, we prove that \( P(n) \) is true for \( n = 8, 9, \) or \( 10 \) by showing how to break classes of these sizes into groups of 4 or 5 students:
      \begin{align*}
8  &= 4 + 4 \\
9  &= 4 + 5 \\
10 &= 5 + 5 
      \end{align*}
Next, we must show that \( P(8), \ldots, P(n) \) imply \( P(n + 1) \) for all \( n \geq 10 \). Thus, we assume that \( P(8), \ldots, P(n) \) are all true and show how to divide up a class of \( n + 1 \) students into groups of 4 or 5. We first form one group of 4 students. Then we can divide the remaining \( n-3 \) students into groups of 4 or 5 by the assumption \( P(n-3) \). This proves \( P(n+1) \), and so the claim holds by induction.
\end{proof}
\end{mdframed}

\begin{parts}
\part This proof contains a critical logical error. (In fact, the claim is false!) Identify the first sentence in the proof that does not follow and explain what went wrong.
  \begin{solution}
    The error is in the second last statement which assumes $P(n-3)$ based on the IH.

    The smallest case for the inductive step is $n=10$, so $P(n-3)$ is $P(7)$ in this case. $P(7)$ is not defined, and in fact cannot be shown to be true.
  \end{solution}
\part Provide a correct strong induction proof that a class with \( n \geq 12 \) students can be divided into groups of 4 or 5.
  \begin{solution}
    For the basis step, we show that $P(12), P(13), P(14),$ and $P(15)$ are true.
    \begin{align*}
      12  &= 4 + 4 + 4 \\
      13  &= 4 + 4 + 5 \\
      14 &= 4 + 5 + 5 \\
      15 &= 5 + 5 + 5
      \end{align*}
      The inductive step is argued similarly as in the question.
  \end{solution}
\end{parts}
\end{questions}
\end{document}
%%% Local Variables:
%%% mode: latex
%%% TeX-master: t
%%% End: